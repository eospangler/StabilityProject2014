\documentclass[12pt]{article}
\pagenumbering{arabic}
\pdfpagewidth 8.5in
\pdfpageheight 11in
\setlength\topmargin{0in}
\setlength\headheight{0in}
\setlength\headsep{0in}
\setlength\textheight{9.0in}
\setlength\textwidth{6.5in}
\setlength\oddsidemargin{0in}
\setlength\evensidemargin{0in}
\setlength\parindent{0.25in}
\setlength\parskip{0.25in}
\parskip 0.0pt
\usepackage[utf8]{inputenc}
\usepackage[english]{babel}
\usepackage{amsmath}
\usepackage{amsfonts}
\usepackage{amssymb}
\usepackage{setspace}
\usepackage{graphicx}
\usepackage{tabularx,ragged2e,booktabs,caption}
\usepackage{float}
\usepackage{dcolumn}
\usepackage{natbib}

\begin{document}  


\begin{titlepage}
	\begin{center}

\vspace*{10 em}
\toprule[1.0pt]\\[0.2cm]
{ \huge \bfseries A Dynamic Model of Political Stability and Dissent \\[0.2cm] }
\bottomrule[1.0pt]

\large
\\[1.2cm]
\emph{Authors:}\\
Ethan \textsc{Spangler}, Washington State University\\
Ben \textsc{Smith}, University of Nebraska-Omaha 



\vfill

% Bottom of the page
{\large \today}
\end{center}
\end{titlepage}






\begin{spacing}{1.5}

%\noindent \textbf{Mad as Hell: a Dynamic Model of Political Dissent and Stability}\\
%Ethan Spangler, PhD Student Washington State University \\
%Ben O. Smith, Assistant Professor University of Nebraska at Omaha


\section{Intro}
The importance of political stability is well established and can affect all aspects of an economy. However, despite its significance, measurement of political stability has remained underdeveloped. As demonstrated by the recent uprisings in the Middle East, Thailand, and Ukraine; large scale political changes are difficult to predict. The three most used measures of political stability are Political Risk Services (PRS), the Business Environment Risk Intelligence Index (BERI), and the Economist Intelligence Unit (EIU). Each of these indexes combine political, financial, and economic factors to assess a nation's political stability (Howell, 1998). The financial and economic factors are predominately based on quantitative data (foreign debt, inflation, GDP per capita, etc) while political factors are far more subjective. 

Political factors are determined and scored by panels of experts (Howell, 1998). These experts are usually former diplomats and scholars. While these teams of experts can be quite large and knowledgeable, it is still a relatively small group of people trying to assess an entire nation and without a theoretical basis for their decisions. Since each index has political factors compose 33-66\% of their measure, if these experts are misinformed the validity of the index could be greatly affected which in turn would affect all research based on these indexes. In a twenty year study, Tetlock (2005) was able to show that forecasts based on expert opinion were only marginally better than random chance. Thus it is the goal of this paper to establish the theoretical basis for a new index of political stability. 

To highlight the issue of why a more robust measure of political stability is needed let us examine some examples and their related political stability analysis. The October 2005 PRS report on Thailand said ``unrest is not expected to threaten general stability, nor intensify to the point of endangering the [Thai Rak Thai Party's] dominant political position...the chances of the TRT being forced from power at any point during the five year forecast period are slim..." Less than a year later a military coup ousted the Prime Minister Shinawatra and outlawed his TRT party, beginning a period of political strife that continues to plague Thailand. The PRS report on Ukraine, published October 2012, stated that ``a repeat of the Orange Revolution...is unlikely." and ``Ukrainians are disillusioned but in general they possess little appetite for protest." Mass protests began in November 2013 and by February 2014 the Yanukovych regime had fallen. The PRS report on Tunisia, published October 2010, called Tunisia an ``oasis of stability" and postulated a 85\% probability that Tunisian dictator Ben Ali would retain power for the next 18 months. By January 2011, mass protests and revolt resulted in the dissolution of the ruling RCD party, the exile of Ben Ali to Saudi Arabia, and the establishment of an interim government. While it may be easy to critique these forecasts with the benefit of hindsight, these examples highlight the difficulty in predicting something as opaque and complex as political stability. A new method of analyzing political stability is needed. 

The bulk of the literature concerning political stability focuses on the factors motivating revolt and revolution. Grossman (1991), Acemoglu and Robinson (2001), and MacCulloch (2001 and 2005) explore the role income inequality has in fomenting revolution. Other approaches concentrate on the role regime type plays (Guttman 2014; Goldstone et al 2010) or government institutions (Acemoglu et al., 2012; Fukuyama, 2014). Diverging from the top down approach, Cameron and Parikh (2000) model the individual choice to join a riot or not. It is felt by the authors of this paper that the focus on revolution, though valuable, does not capture the entirety of the situation. Revolution is the end point on a long process. A nation can exist for a long time in a state of low political stability but not descend into revolt. Thus, in a departure from the literature we will be focusing on the decisions and dynamics that precipitate a revolution. 

Our model extends the work of previous scholars by focusing on the dynamic interactions of a non-altruistic government and citizenry. We cannot predict that spark that will ignite a revolution, but we be able to establish a framework for measuring the environment that allows a spark to turn into a revolution. Through simulation we determine the maximum exogenous shock that does not result in a revolution. This translates to a probability of failure of the state. The end result is a framework which can be used to develop a new index of political stability. Whereas current measures have no clear interpretation, ours would have the defined statistical interpretation of a probability a given event occurring. This makes it a far more robust and useful measure.    

\section{Theoretical Model}

%Incorporate the work of Gordon Tullock, the Social Dilemma, and all the other Public Choice lit stuff. 

%The Paradox of Rebellion will also matter

%Our paper builds off of the work of previous literature in the area of public choice. 


%We avoid the issue of the collective action problem by not focusing on the revolution itself, but the events leading up to it. 

%As (The North Korea Paper) discuss extensively, the collective action problem of revolution is not easily sorted theoretically. We artfully avoid these issues by focusing on events precipitating a revolution. 

%We also build a dynamic structure. 

%However, we step away from the traditional game theory models and instead build a dynamic model. 


The model begins in period 0. The catalyst the model is the initial amount total dissent, $D_0$. This $D_0$ can be thought of as either anarchists or remnants of the previous regime, their origin doesn't really matter just that it exists. A government then forms to manage and organize affairs, setting initial policy choices. The government maximizes political stability by choosing its level of public and security spending, subject to a resource constraint and the level of discontent from the citizenry. In turn, each member of the public chooses their level of dissent based on their own preferences and the probability of punishment.
 
\subsection{Individual's Problem}
Initial framing of the individual begins in the general sense as a household problem, 

\begin{equation}
	U(c,l) \text{ s.t. } M
\end{equation}

\noindent where utility, $U(c,l)$, is a function of consumption and leisure subject to an income constraint, $M$. The individual finds a solution for themselves. This represents the aspects of an individual's life that they have control over. However, the world being the imperfect place that it is, there are some things that an individual can't control.  

The story is that in their daily lives, individuals face societal problems. It could be as mundane as a long line at the DMV. Perhaps they encountered a corrupt police officer. Another scenario could potential involve members of a different group harassing the individual. Maybe they saw a story about a wealthy, well-connected elite dodging criminal charges. Whatever the particular issue, it is a problem that the individual cannot directly solve themselves but it has negatively affected their life. These are societal issues that only a government could properly address. However, for whatever reason the government is unable/unwilling to address these problems completely to the satisfaction of the individual. The incongruousness between an individual's expectations of what a government should do and the reality of what a government does do is the origin of dissent. So without any direct power to alter their situation the individual does what people usually do in such situations, they dissent. The individual uses a portion of what would be their leisure and instead uses it to dissent, which provides a cathartic release and gives the individual some level of utility with how much extra utility dictated by parameters. 

\vspace{.5 em}
\noindent Individual's Problem:
\begin{equation}
{\underset{d_{i,t}}{\text{max }}}  U_{i,t}= \frac{{d_{i,t}}^{{x}_i}}{g_t * E} - P \left(\frac{D_{t-1}}{S_t}\Bigg|\sigma \right)d_{i,t}^A
\end{equation}

$ d_{i,t} $ is the amount individual $i$ dissents in period $t$ and is treated as a level variable wherein each period an individual choices their level of dissent based on the situation they find themselves in and their own preferences. We assume that dissent takes only non-negative values, $d_{i,t}\geq0$. Different levels of $d_{i,t}$ have different interpretations. An individual with $d_{i,t}=0$ is interpreted as having no dissent, this person is either content with the government or too scared to dissent. Alternatively a positive value of $d_{i,t}$ can be interpreted as being more active; on the low end, $d_{i,t}>0$, could be contacting government representatives, attending open forums, voting for opposition parties, or kvetching about politics with colleagues. At the extreme end, $d_{i,t}>>0$, dissent takes on a more revolutionary bent: protests, riots, molotovs. Since individuals dissent during time they would otherwise be using for leisure there is a maximum amount a single individual can dissent for a given period, $d_{i,t}^\star$.  

$x_i$ is the exponent that represents an individual's level of activism with respect to dissenting. We assume $x_i \sim logN(0,1)$, the Log-Normal PDF. Different values of $x_i$ have different interpretations. An individual with a $x_i$ on the left of the distribution could be seen an individual with high anxiety when it comes to dissenting. At the mean, these individuals would be comfortable discussing their frustrations among close peers (``Dinner Party Dissenters") but more active actions is unlikely. As we move to the right of the mean each higher value of $x_i$ represents a more activist role the individual would be willing to undertake. Since we have made an assumption on the distribution of activism, this allows us to determine how much dissent there will be in a country each period for given government policy choices.    

$E$ and $g_t$ reduce the amount of utility and individual gets from dissenting. $E$ is an exogenous parameter assessing an individual's quality of life each period. By environment we mean an individual's living situation within a country: unemployment, healthcare, safety, equality, etc. $E$ allows us to weigh the validity of dissent for comparisons across countries. Dissent in a country where conditions are good (``first world problems")  are far less meaningful than in countries with more difficult situations. By weighing dissent based on the circumstances of a country, we are better able to establish an index of political stability that works for both developed and developing countries. $g_t$ is per capita government spending($g_t=\frac{G_t}{N_t}$ where $N_t$ is the population) and is treated as given in the individual's problem. $g_t$ represents how much the government services are given to each citizen. Since this is a pure public good, the government cannot restrict who receives this, thus the individual receives this benefit regardless of their level of dissent. 

%Anti-hiprocacy assumption, that the utility from dissent is diminished the more the government gives. 

%A Pareto Distribution might be more computationally effective. 


$P \left(\frac{D_{t-1}}{S_t}\Big|\sigma \right)$ is a the probability of being caught dissenting against the government and it it follows the log Normal PDF. $S_t$ government resources spent on quelling dissent and is treated as given in the individual's problem. $D_{t-1}$ is the preexisting total dissent. $\sigma$ represent policing effectiveness. For low $\sigma$ the police are extremely prompt at arresting those that dissent beyond acceptable limits. For higher $\sigma$ it probability of being arrested becomes more spread out. For practical purposes $\sigma$ is the ratio of $\frac{\text{Total Reported Crime}}{\text{Crimes Cleared}}$. While not perfect, it does provide some insight into police effectiveness. $A$ this is the punishment parameter for dissenting and it structured such that $A>1$. Since the punishment should fit the crime, an individual is punished based on their level of dissent. An assumption with $P \left(\frac{D_{t-1}}{S_t}\Big|\sigma \right)$ is that the amount an individual is dissent does not affect their probability of being caught for dissenting. The reasoning is that the amount of dissent a single individual could produce is insignificant within an entire nation ($d<<D$), a drop in the ocean so to speak. That said, if an individual is caught they will be punished on the amount they have dissented.   

%I could include a graph here to sell it better. 

\subsection{Government's Problem} 

In our view there is no reason to think a government exclusively seeks to maximizes social welfare. Evidence abounds with leaders less than sympathetic towards the plights of their constituents: North Korea, Syria, Congress. In fact, the only thing that leaders seem to care about is retaining power, obtaining some hidden benefit that is contingent upon sustained stability. Thus it then becomes the goal of government to maximize political stability. 

The Government's problems begins with the with the Leader's problem. As stated previously stated, leaders receive some benefit from holding power. The leaders could be taking funds from the treasury, receiving a portion of resource export profits, or have firms they're heavily invested in receive lots of lucrative no-bid government contracts. Whatever the case, the leaders gets this benefit (and the utils) only if they are in power. So leaders seeks to maximize their own utility subject to political stability ($\Lambda_t$) remaining positive.

\vspace{.5 em}
\noindent Leaders' Problem:
\begin{equation}
	 \text{max } U_t(B_t) \begin{cases}
		>0 \text{ if } \Lambda_t > 0 \\
		= 0 \text{ otherwise} 	
	\end{cases}
\end{equation} 

\noindent where $B_t$ is the benefit from holding power that the leader takes. However, this benefit is not costless and it diminishes the amount of resources the Government has to use to maintain stability, $B_t=R_t-C_t$ where $R_t$ is government resource and $C_t$ is the cost of maintaining stability. Thus the leader lines their pockets but must be careful not to take too much or they'll facilitate their own demise. With the leader's problem established, we move on to the task of maintaining stability.   
\vspace{.5 em}

\noindent Government's Problem:
\begin{equation}
{\underset{G_t,S_t}{\text{max }}} \sum\limits_{t=1}^{T^*} \beta^t {\Lambda}_t = \sum\limits_{t=1}^{T^*} \beta^t\left(\frac{G_t}{D_{t-1}}-\Phi\right)^\alpha \left(\frac{S_t}{D_{t-1}}-\Omega\right)^\gamma   \text{s.t. } G_t+S_t=C_t
\end{equation}

It is the goal of the government to maximize it's stability over time. $G_t$ is government spending on government services. Basically $G_t$ provides all the things we expect a functioning government to provide (schools, hospitals, DMVs, etc). $S_t$ is security spending. If $G_t$ is the carrot, $S_t$ is the stick. These are resources that the government uses to suppress and control the people. 

$\Phi$ and $\Omega$, are the anarchy conditions, representing the minimum amount of government and security spending needed order for the government to function. If either of these minimum conditions are not met for whatever reason, such as an exogenous shock, stability goes to zero and you have a revolution. This is why the problem is maximized to $T^*$ not $\infty$, since all governments eventually fail. Should $\Lambda_t=0$, that would signify that the state has broken down and a revolution has taken place, resetting the system. $\beta^t$ is the government's discount factor for stability. 

$\alpha$ and $\gamma $  are parameters that represent how much a government favors using $G_t$ and $S_t$ respectively. One would expect a democracy to favor $G$ over $S$ ($\alpha > \gamma$) but the opposite would hold for a totalitarian regime ($\alpha < \gamma $). $G_t+S_t=C_t$ is the Government resource constraint. No government can spend unlimited amounts of resources, so they face a constraint. $C_t$ is government resources available to the decision maker each period. 

$D_{t-1}$ is total dissent. Dissent reduces the effectiveness of government policy and in turn makes it difficult for the government is retain power. As shown, the government reacts to previous period of dissent whereas the individual reacts to the current period. The result is that the government simply cannot keep up with the situation on the ground. This bureaucratic delay is the source of instability. When $D_t \approx D_{t-1}$ the instability is small, but when $D_t \not\approx D_{t-1}$ instability increases. Continued incongruences could result in the situation spiraling out of control. 

%Expand here. 

%I need a fictional unit for Dissent..."punks", "molotovs", erises    

\section{Solving the Model}

This is a two stage dynamic interaction between the government and the public. In the initial period the government sets initial policy $G_1$ and $S_1$ based on some starting value of $D_0$, the initial total dissent. $D_0$ can be interpreted a handful of ways: they could be from a group of anarchists (they just hate all government) or maybe the losers in the contest that established the current government. Whatever their origins we just need an initial level of dissent to be greater than 0. Since no government has ever formed without some form of acrimony, this is not an unreasonable assumption. We can solve for an equilibrium through backwards induction, the goal being to determine dynamics of $D_t$. So we'll start with the individuals problem. 

\subsection{Individual's solution}

%It must be stated that there is no analytic solution to an individuals demand for dissent, which we will call $\overline{d}_{i,t}$. Demand for dissent must be numerically determined in simulations, which is discussed in the next section. 

The first step in determining individual demand for dissent is first discussing the scenarios which do and do not facilitate dissent. This requires examining the first order and second order conditions.  

\noindent Differentiating the individual problem with respect to $d_{i,t}$ we obtain the FOC: 
\begin{equation}
\frac{dU}{dd_{i,t}} = x_i \frac{{d_{i,t}}^{x_i -1}}{E g_t} - P\left(\frac{D_{t-1}}{S_t}\Big|\sigma \right)Ad_{i,t}^{A-1}  
\end{equation}

\noindent Differentiating again with respect to $d_{i,t}$ we obtain the SOC: 

\begin{equation}
\frac{d^2U}{dd_{i,t}^2}=(x_i -1) x_i \frac{{d_{i,t}}^{x_i -1}}{E g_t} - P\left(\frac{D_{t-1}}{S_t}\Big|\sigma \right)(A-1)Ad_{i,t}^{A-2}  
\end{equation}

\noindent For the $\frac{dU}{dd_{i,t}}$ and $\frac{d^2U}{dd_{i,t}^2}$ Let:    
\begin{center}
$\underbrace{ x_i \frac{{d_{i,t}}^{x_i -1}}{E g_t}}_\textrm{H} - \underbrace{P\left(\frac{D_{t-1}}{S_t}\Big|\sigma \right)Ad_{i,t}^{A-1}}_\textrm{I}$  	
\end{center}

\noindent and: 

\begin{center}
$\underbrace{(x_i -1) x_i \frac{{d_{i,t}}^{x_i -2}}{E g_t}}}_\textrm{J} - \underbrace{
P\left(\frac{D_{t-1}}{S_t}\Big|\sigma \right)(A-1)Ad_{i,t}^{A-2}}_\textrm{K}$  
\end{center}

\noindent Then the various scenarios are: 
\begin{enumerate}
\item If $x_i<0$, then $\frac{dU}{dd_{i,t}} < 0$ and $d_{i,t}=0$. 
\item If $x_i>0$ and $H<I$, then $\frac{dU}{dd_{i,t}} < 0$ and $d_{i,t}=0$. 
\item If $1>x_i>0$ and $H>I$; then utility is maximized when $\frac{dU}{dd_{i,t}}=0$, $\frac{d^2U}{dd_{i,t}^2}<0$, and $d_{i,t}>0$.  
\item If $x_i>1$, $H>I$, and $J<K$; then utility is maximized when $\frac{dU}{dd_{i,t}} = 0$, $\frac{d^2U}{dd_{i,t}^2}<0$, and $d_{i,t}>0$. 
\item If $x_i>1$, $H>I$, and $J>K$, then $\frac{dU}{dd_{i,t}} > 0$, $\frac{d^2U}{dd_{i,t}^2}>0$, and $d_{i,t}=d_{i,t}^\star $. 
\end{enumerate}

Scenarios 1 through 4 pose very little to be concerned about, providing a nice distribution of dissenters amongst the populace. The $\text{5}^{\text{th}}$, however, requires a bit more explanation. Given the previously stated income constraint, $M$, and by extension the maximum possible amount of dissent, $d_{i,t}^\star$; during situations in which $\frac{dU}{dd_{i,t}}$ and $\frac{d^2U}{dd_{i,t}^2}$ are positive the individual would merely dissent the maximum amount possible.  

With our various scenarios outlined we can now move on to the process of deriving individual demand for dissent, which is a simple algebraic process. 

\vspace{1 em}
\noindent Begin with the first order condition: 

\begin{equation}
	\begin{aligned}
\frac{dU}{dd_{i,t}}=& 0\\ 
x_i \frac{{d_{i,t}}^{x_i -1}}{g_t E} - P\left(\frac{D_{t-1}}{S_t}\Big|\sigma \right)Ad_{i,t}^{A-1}=& 0 \\
	x_i \frac{{d_{i,t}}^{x_i -1}}{g_t E} =& P\left(\frac{D_{t-1}}{S_t}\Big|\sigma \right)Ad_{i,t}^{A-1}\\
		d_{i,t}=& \left(\frac{g_tEP(S_t,D_{t-1})A}{x_i} \right)^{\frac{1}{x_i -A}}\\ 	
	\end{aligned}
\end{equation}

We shall all this solutions $\overline{d}_{i,t}$ and use it to determine the expected total dissent, $D_t$. 

%To avoid this in simulation, if a random draw of $x_i$ is one such that it makes the $SOC>0$ it will be automatically lowered to the maximal value, $x_{max}$, where $SOC<0$ holds. This constrains should not prove too onerous.       
   
\subsection{Total Dissent}
 
To find the expected total dissent, we integrate $\overline{d}_{i,t}$ times the log normal PDF, $f(x)$, across the relevant range of $x_i$, $0$ to $x_{max}$, will give the average amount of dissent per capita in a country. Multiplying this by the population, $N_t$, gives the expected total dissent, $D_t$.

\begin{equation}
	\begin{aligned}
D_t=& N_t \int_{0}^{x_{max}} \bar{d}_{i,t} f(x) dx \\	
D_t	=& N_t \int_{0}^{x_{max}} \bar{d}_{i,t} \frac{1}{x\sigma \sqrt{2\pi}}exp  \left( -\frac{(lnx-\mu)^2}{2\sigma^2} \right)  dx \\	
	\end{aligned}
\end{equation}

As you may be able to tell, a solution for total dissent can not be analytical determine so we must rely on numerical simulation which will be discussed in the next section. 


\subsection{Government Solution}

To solve the government's problem we need a Lagrangian.
\begin{equation}
\mathcal{L} = \left(\frac{G_t}{ D_{t-1}}-\Phi\right)^\alpha \left(\frac{S_t}{ D_{t-1}}-\Omega\right)^\gamma  +\sigma[R_t-G_t-S_t] 
\end{equation}
Take the FOCs and simplify.

\begin{equation}
    \begin{aligned}
        G_t\text{: } \alpha \left(\frac{G_t}{ D_{t-1}}-\Phi\right)^{\alpha-1} \frac{1}{ D_{t-1}} \left(\frac{S_t}{D_{t-1}}-\Omega\right)^\gamma  -\lambda &=0  \\
S_t\text{: } \gamma  \left(\frac{G_t}{ D_{t-1}}-\Phi\right)^{\alpha} \frac{1}{D_{t-1}} \left(\frac{S_t}{D_{t-1}}-\Omega\right)^{\gamma -1} -\lambda &=0 \\
\alpha \left(\frac{G_t}{D_{t-1}}-\Phi\right)^{\alpha-1} \frac{1}{ D_{t-1}} \left(\frac{S_t}{D_{t-1}}-\Omega\right)^\gamma  &= \gamma  \left(\frac{G_t}{ D_{t-1}}-\Phi\right)^{\alpha} \frac{1}{ D_{t-1}} \left(\frac{S_t}{ D_{t-1}}-\Omega\right)^{\gamma -1} \\
\alpha \left(\frac{S_t}{ D_{t-1}}-\Omega \right) &= \gamma  \left( \frac{G_t}{ D_{t-1}}-\Phi \right) \\
G_t&= D_{t-1}\left[\frac{\alpha}{\gamma } \left(\frac{S_t}{ D_{t-1}} -\Omega \right)+\Phi \right]
    \end{aligned}
\end{equation}


\noindent Plug this into the government resource constraint. 


\begin{equation}
    \begin{aligned}
        S_t+G_t&=R_t \\
        S_t+  D_{t-1}\left[\frac{\alpha}{\gamma } \left(\frac{S_t}{ D_{t-1}} -\Omega \right) +\Phi \right]  &= R \\
S_t+ \frac{\alpha S_t}{\gamma } -\frac{\alpha  D_{t-1} \Omega}{\gamma } +D_{t-1}\Phi &=R_t \\
S_t\left(1+\frac{\alpha}{\gamma }\right) &= R_t+ \frac{\alpha D_{t-1} \Omega}{\gamma } - D_{t-1}\Phi \\
S_t\left(\frac{\gamma  + \alpha}{\gamma }\right) &= R_t - D_{t-1} \left(\Phi - \frac{\alpha  \Omega}{\gamma } \right) 
    \end{aligned}
\end{equation}


\noindent Solve for $S_t$ and we get
\begin{equation}
S_t=\frac{\gamma }{\gamma  +\alpha} \left[ R_t - D_{t-1} \left(\Phi - \frac{\alpha}{\gamma }\Omega \right) \right]
\end{equation}
\noindent This is the government's security equation. We repeat the same steps to find $G_t$. 
\begin{equation}
G_t=\frac{\alpha}{\gamma  +\alpha} \left[ R_t - D_{t-1} \left(\Omega - \frac{\gamma }{\alpha}\Phi \right) \right]
\end{equation}

\section{Simulation} 

Given that a complete analytic solution for the theoretical model is not possibly, numerical methods of simulation must be employed. To do this we developed eight country archetypes to test. While there are endless possible combination of the parameters, these 8 were chosen because they represent a broad range of characteristics and provide the most theoretical insight as to how differing parameter values affect stability. The values used in the simulations are based on extrapolations of real world data but are refined to fit within the confines of the simulation. The explanation of each simulation country includes a list of which real countries which were used as a basis for forming the parameters values. 

Table 1 details the origin of the values used for the parameters in the simulations. Preference($\alpha$ and $\gamma$) and the anarchy conditions ($\Omega$ and $\Phi$) are based on estimates from sample countries. Government resources,$R$, are based on GDP figures. $\sigma$ represents the effectiveness of security forces in suppressing dissent. The value of $\sigma$ was based on the ratio of total reported crime over total cleared\footnote{Cleared meaning that there was a resulting arrest. This was used over merely counting arrest because some criminals commit multiple crimes.} crime. In 2014 there were over 9.4 million violent (murder, assault, rape) and property (theft, larceny, vandalism) crimes in the US. For these crime and of these 2.2 million were cleared. The ratio of these provides an approximate $\sigma$ value of 4 which is what was used as a starting value in the simulations. Due to data limitations regarding crime rates around the world, this initial value of $\sigma$ was applied to the other simulation parameters with minor changes to create theoretical variation. $A$ is based on the Political Rights scores from Freedom House on a scale of 1\footnote{For computational reasons explained earlier a minimum value of 1.1 is used for simulation.} being the most free and 7 being the least. $E$ is based on the the UN's Human Development Index. Population is held constant between countries. Simulations were conducted using Mathematica. For all simulations only 100 periods were analyzed. This was done because the authors are more concerned with the short-term fluctuations in stability. The following is a brief synopsis of each simulated country:  

%based off of theoretical constructs (cite people)

%Consult the old Developement Lit: 
\begin{itemize}
	\item \textbf{Freedonia:} meant to represent the developed, socialist, and soft-power focused Western states. As such the parameters features a wealthy and efficient government, a preference for using government services to quell dissent, and low punishment for those that are caught dissenting, and a high quality of life for the citizens. Countries used for parameter basis: Canada, Japan, Western Europe, and Scandinavia. 
	\item \textbf{`Merika:} very similar to Freedonia in terms of preferences but slightly more disposed to using force to suppress dissent and slightly reduced quality of life.  Countries used for parameter basis: Australia, South Korea, UK, and US. 
	\item \textbf{Kleptopia:} the antithesis of Freedonia. Kleptopia is a wealthy developed state but instead of a preference for using government services to maintain order, Kleptopia prefers the brute force method while also having a lower quality of life for its citizens.  Country used for parameter basis: Russia.  
	\item \textbf{Cathay:} more balanced in preferences but with a slight list towards suppression. Is representative of states on the cusp of political and economic development. Punishment for dissenting is still quite high but quality of life is decent.  Countries used for parameter basis: Brazil, China, India.   
	\item \textbf{Rentistan:} typical wealthy rentier state. Has equal preferences for both government services and suppression but also has low quality of life for it's citizens and is prone to corruption and inefficiency.  Countries used for parameter estimates: Nigeria, Qatar.\footnote{There was insufficient data from other resource rich states.} 
	\item \textbf{Develpolus:} the prototypical developing nation. Has equal preferences in regards to methods of maintaining order but is poor, has a low quality of life, and the government is either inefficient, corrupt, or both; leading to higher fixed costs. Countries used for parameter basis: Jordan, Kenya, Nepal.  	
	\item \textbf{Bellicostia:} similar to Kleptopia but lacks both the resources and skills. Resources are limited but are used primarily on suppression. Indicative of small autocratic states.  Countries used for parameter estimates: Iran, Pakistan, Syria. 
	\item \textbf{Hippieberg:} similar to Freedonia but lacks both the resources and skills. There is a preference for using government services over suppression but has few resources. Countries used for parameter basis: Bhutan, Uruguay, the Baltic states.  
\end{itemize}

\begin{table}[]
\centering
\begin{tiny}
\caption{Parameter Basis}
\begin{tabular}{llll}
\toprule
\textbf{Label} & \textbf{Variable} & \textbf{Basis} & \textbf{Source} \\ \hline
\textbf{$\alpha$} & preference for G & estimated &  \\
\textbf{$\gamma$} & preference for S & estimated &  \\
\textbf{$\Omega$} & G fixed costs & estimated &  \\
\textbf{$\Phi$} & S fixed costs & estimated &  \\
\textbf{$R$} & Government Resources & GDP & World Bank \\
\textbf{$\sigma$} & Security Effectiveness & $\frac{\text{Total Reported Crime}}{\text{Cleared Crimes}}$ & US FBI Crime Statistics \\
\textbf{$A$} & Dissent Punishment & Political Freedom Score & Freedom House \\
\textbf{$E$} & Quality of Life & Human Development Index & UN \\ \hline
\end{tabular}
\end{tiny}
\end{table}

\begin{table}[]
\centering
\begin{tiny}
\caption{Simulation Parameters}
\begin{tabular}{lcccccccc}
\toprule
\multicolumn{1}{l}{} & \textbf{Freedonia} & \textbf{‘Merika} & \textbf{Kleptopia} & \textbf{Cathay} & \textbf{Rentistan} & \textbf{Develpolus} & \textbf{Bellicostia} & \textbf{Hippieberg} \\ \hline
\textit{$\alpha$} & 0.9 & 0.8 & 0.1 & 0.4 & 0.5 & 0.5 & 0.1 & 0.9 \\
\textit{$\gamma$} & 0.1 & 0.2 & 0.9 & 0.6 & 0.5 & 0.5 & 0.9 & 0.1 \\
\textit{R} & 350,000,000 & 350,000,000 & 350,000,000 & 150,000,000 & 300,000,000 & 100,000,000 & 50,000,000 & 50,000,000 \\
\textit{$\Omega$} & 35,000,000 & 35,000,000 & 35,000,000 & 15,000,000 & 60,000,000 & 20,000.00 & 10,000,000 & 10,000,000 \\
\textit{$\Phi$} & 35,000,000 & 35,000,000 & 35,000,000 & 15,000,000 & 60,000,000 & 20,000,000 & 10,000,000 & 10,000,000 \\
\textit{$\sigma$} & 3 & 3 & 3 & 6 & 9 & 9 & 5 & 5 \\
\textit{A} & 1.1 & 1.1 & 6 & 6 & 5 & 5 & 6 & 1.1 \\
\textit{E} & 0.9 & 0.85 & 0.7 & 0.7 & 0.7 & 0.5 & 0.6 & 0.6 \\
\textit{N} & 10,000 & 10,000 & 10,000 & 10,000 & 10,000 & 10,000 & 10,000 & 10,000 \\ \hline
\end{tabular}
\end{tiny}
\end{table}

\subsection{Initial Dissent}

The first round of simulations focus on how states with different parameters adapt to different levels of initial dissent, $D_0$. Low, medium, and high levels of initial dissent where applied to each country and the simulation was run for 100 periods. The plotted stability can be seen in Figure (NUMBER!)

%\begin{figure}[htb]
 % \centerfloat{
  % \includegraphics[width=\textwidth]{FIGURENAME}}
%\end{figure}

The plots adhere to our earlier point that countries can be stabile in their instability since every country that could achieve stability in the initial stage was able to perpetuate itself. 


Analysis of the plots. 

\subsection{Shocks}

We also wanted to observe how robust various countries are to exogenous shocks. 

At period 50, an exogenous shock was put into the system and remained for the next 50 periods of the run. 

government shocks and individual shocks 

\begin{itemize}
	\item \textbf{$R$:} the government resources are severely cut. The president emptied the treasury and fled to a non-extraditing country. 
	\item \textbf{$\Omega,\Phi$:} the government is hit with new levels of corruption and more resources must be spent on the Anarchy parameters. 
	\item \textbf{$E$:} the quality of life suddenly drops. Mass joblessness, public health crisis, environmental degradation. 
	\item \textbf{$\lambda$:} the government is more effective at catching people that dissent 
	\item \textbf{$A$:} the government is far more harsh on those it catches dissenting 
\end{itemize}







\section{Conclusion}


(Work in progress)

%It needs to be said that this models only applies to cases on internal strife and does not take into account the role of outside agitators. 



\end{spacing}


\pagebreak

\bibliographystyle{apalike}
\bibliography{Stability_Bibli}

\nocite{*}





\end{document}
